
\title{Introduktion}
\section{Introduktion}


\begin{frame}[fragile=singleslide]
	\frametitle{Välkommen}
	\begin{itemize}
		\item Syftet med \emph{dod}-delen?
				\begin{itemize}
					\item Lär dig grundläggande verktyg.
					\item Introduktion inför kommande kurser.
				\end{itemize}
		\item Vem läser kursen?
		      \begin{itemize}
				\item C-programmet.
				\item D-programmet. 
		      \end{itemize}
		\item Vad innehåller kursen?
		      \begin{itemize}
			      \item Fyra ämnen, med vardera en föreläsning och en laboration
			      \begin{enumerate}
					\item Unix och Linux
					\item \LaTeX
					\item Git och Github
					\item Maskinkod
				  \end{enumerate}
		      \end{itemize}
	\end{itemize}
\end{frame}


\begin{frame}[fragile=singleslide]
	\frametitle{Schema och gruppindelning}

	\begin{itemize}
		\item Gruppindelning
		      \begin{itemize}
			      \item Gruppindelning, se Canvas.
			      \item Kapten Alloc \url{https://fileadmin.cs.lth.se/pgk/kaptenalloc/}
		      \end{itemize}
		\item Schema
		\item Föreläsning alltid samma tid (tisdag 15:15), E:A
		\item Laborationer
		      \begin{itemize}
			      \item En labb per vecka, start imorgon!
			      \item Föreberedelseuppgifter, kontrollfrågor.
			      \item Laboration 1, Linux (Första laborationen på torsdag!)
			      \item Laboration 2. \LaTeX
			      \item Laboration 3, Git
			      \item Laboration 4, Maskinkod
		      \end{itemize}
	\end{itemize}
\end{frame}



\begin{frame}[fragile=singleslide]
	\frametitle{Kurshemsidan och Github}

		{\bf Den öppna kurshemsidan:}

		\smallskip
		\small\url{https://lunduniversity.github.io/pgk/}
		
		
		\bigskip
		
		
		{\bf Allt kursmaterial finns tillgängligt på GitHub:}

		\smallskip
		\small\url{https://github.com/lunduniversity/introprog-computer-intro}
		
		\smallskip
		\begin{itemize}
			\item Källkod och material för kursen.
			\item Lämna issues om ni upptäcker problem eller vill föreslå förbättringar.
			\item Pull requests är alltid välkomna!
		\end{itemize}
\end{frame}

\begin{frame}[fragile=singleslide]
	\frametitle{Vid frågor}

	\begin{itemize}
		\item Ni kan alltid maila mig: \url{mattias.nordahl@cs.lth.se}
		\item Discord. Hitta invite-länk på Canvas-sidan.
	\end{itemize}
\end{frame}