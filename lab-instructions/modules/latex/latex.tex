
\section{Laboration \arabic{section} --- \LaTeX}

\emph{Mål:} Du ska lära dig grunderna i {\LaTeX} och tillämpa dina kunskaper på ett exempel.


\subsection*{Obligatoriska förberedelser (hemarbete)}
\begin{Hemarbete}\firmlist
	\item Titta igenom föreläsningsbilderna till föreläsningen om \LaTeX.
	\item Läs igenom kompendiet \emph{Att skriva rapporter med \LaTeX}, åtminstone så mycket så att du blir bekant med vad man kan göra med \LaTeX. Du behöver inte försöka memorera alla detaljer.
	\item \label{hem:latexuppg} Med början två sidor fram finns ett exempel på en rapport som är producerad med \LaTeX. Studera rapporten och försök komma på vilka kommandon som behövs för att få texten att se ut som den gör. Markera i rapporten, eller i ett separat textdokument, vilka kommandon du behöver använda för att efterlikna rapporten.
\end{Hemarbete}

\Förberedelser
\begin{Kontrollfragor}
	\item Vad är skillnaden mellan {\LaTeX} och vanliga textredigerare, som MS Word?
	\item Vilka fördelar finns med att använda {\LaTeX} för dokument?
	\item Vad är ett {\LaTeX}-kommando och hur skrivs det i en dokumentfil?
	\item Vad är skillnaden mellan ett kommando och en omgivning (environment)?
	\item Hur skapar man rubriker och underrubriker i {\LaTeX}?
	\item Hur får man fet eller kursiv text i {\LaTeX}?
	\item Vad betyder tecket \$ i {\LaTeX}?
	\item Hur skriver du tecknet \$ i {\LaTeX}, utan att det tolkas som ett kommando?
	\item {\LaTeX} är helt textbaserat. Vad är processen för att producera det slutgiltiga, formaterade dokumentet?
	\item Vad är en paketfil i {\LaTeX} och hur inkluderar man den i dokumentet?
	\item Det finns tre typer av listor i {\LaTeX}: punktlistor, numrerade listor och beskrivningslistor. Hur skapar du var och en av dessa?
	\item Hur numreras figurer och tabeller i {\LaTeX}?
	\item Hur refererar du till en figur eller tabell i texten?

\end{Kontrollfragor}

\newpage

\subsection*{Datorarbete}

\textbf{Notis}: Du får avända valfri \LaTeX-editor, t.ex. VS Code, TexmakerI instruktionerna nedan föreslår vi att ni använder programmet Texmaker för att arbeta med \LaTeX, men ni får lov att använda vilken editor ni vill. Det går också bra att använda onlineverktyg, så som Overleaf.

\begin{Datorarbete}
	\item I mappen med labbfiler som du laddade ned i labb 1 finns också en \file{latex}-katalog, som innehåller allt du behöver för att återskapa rapporten på nästa sida. Den finns här:
	
	\url{https://fileadmin.cs.lth.se/pgk/dod-lab-material.zip}
	
	\begin{itemize}
		\item Filen \file{sort\_scala.tex} är en mall för \LaTeX-dokumentet, men utan innehåll.
		\item I filen \file{oformaterad\_text.txt} finns den råa texten utan formattering.
	\end{itemize}
	\item Starta din föredragna \LaTeX-editor och öppna filen \file{sort\_scala.tex}, som är en mall för \LaTeX-dokumentet, med alla paket inkluderade (se kommentarerna i filen). Kopiera in den råa texten från \file{oformaterad\_text.txt} mellan \verb/\begin{document}/ och \verb/\end{document}/.
	\item Återskapa den färdiga rapporten på nästa sida. Lägg in lämpliga \LaTeX-kommandon i filen så att rapporten får (åtminstone ungefär) samma utseende.
	\begin{itemize}
		\item Arbeta stegvis -- ändra lite, bygg/kompilera och titta på resultatet, ändra lite till, osv.
		\item Notera att vissa saker i den råa texten kan leda till att kompileringen misslyckas, t.ex. specialtecken eller matematiska symboler som \(\cdot\) (multiplikation).
		\item Bilderna som ska inkluderas i dokumentet finns också bland de nedladdade labbfilerna, samt programkoden för kodlistningen i rapporten.
		\item Tänk på att vissa detaljer i den råa texten (t.ex. kapitel-, sid- och figurnummer) behöver tas bort, eftersom de genereras automatiskt av \LaTeX.
	\end{itemize}

    \smallskip

    \noindent\textbf{Tips:}
    \begin{itemize}
        \item Kolla vilka paket som är inkluderade i mallen \file{sort\_scala.tex}.
        \item Ta hjälp av \LaTeX-häftet och/eller online-resurser för att hitta rätt kommandon.
        \item Använd etiketter (\verb/\label{etikett}/) och referenser (\verb/\ref{etikett}/) för att referera till figurer, tabeller och sektioner. \LaTeX\ numrerar automatiskt dessa åt dig. Skriv dem inte manuellt.
        \item Den råa texten kan innehålla radbrytningar och bindestreck. Dessa ska inte ingå i källkoden, utan låt \LaTeX\ sköta radbrytningarna automatiskt där det är lämpligt.
        \item Din lösning behöver inte vara \emph{exakt} likadan som den färdiga rapporten. Om du kör fast, fråga labbhandledaren om hjälp.
        \item I appendix numrerar man kapitel med bokstäver (A, B, C, ...). Det görs med kommandot \verb/\appendix/. Rubriker som kommer efter det kommandot numreras med bokstäver.
		\item Ibland kan det vara bra att kommentera ut delar av koden med \verb/%/ i början av raden, för att enklare kunna felsöka.
    \end{itemize}

	\item Om du har tid: prova sådana möjligheter i \LaTeX\ som du inte har behövt använda tidigare: listor av olika slag, innehållsförteckning, mera avancerade formler, osv.
	\item Kolla gärna på lösningen \emph{efter labben}, som finns på GitHub\footnote{Gå till dod-repot: \url{https://github.com/lunduniversity/introprog-computer-intro}\\
	och titta på: \url{lab-instructions/modules/latex/newex/sort_scala_losning.tex}}, men försök att inte titta på den förrän du har gjort så mycket du kan på egen hand.
\end{Datorarbete}

\includepdf[pages=-]{modules/latex/newex/sort_scala_losning}