\documentclass[a4paper]{memoir}

% Swedish.
\usepackage[T1]{fontenc}
\usepackage[swedish]{babel}

\usepackage{selthcscourseplan}

%---------------------------------------------------------------

% Utilities.
\usepackage{microtype}
\usepackage{url}

%*****************************************************************
\begin{document}

\chapter*{Datorer och datoranvändning}
\courseinfo{EDAA60 för D1, EITA65 för C1}{läsperiod 1 2024/25, 3 högskolepoäng.}

\section*{Allmänt}

\begin{Items}
    \item{}
    OBS! Denna information gäller både dig som läser på Datateknikprogrammet och går kursen EDAA60 Datorer och datoranvändning,
    samt dig som läser Informations- och kommunikationsprogrammet och går kursen EITA65 Digitalisering -- realisering och systemdesign med användarperspektiv.

    \item[Välkommen!]
    I kursen Datorer och datoranvändning får du en översikt över datatekniska begrepp och praktiska färdigheter som du kommer att ha användning av under dina studier. Kursen läses av studenter från C- och D-programmen. För C-studenter ingår den som en del av kursen EITA65.

    \item[Kursansvarig]
    Mattias Nordahl, rum E2184, E-huset 2:a våningen.\\
    E-post: \url{mattias.nordahl@cs.lth.se}

    \item[Administratör]
    Ulrika Templing, rum E2179, E-huset 2:a våningen. Expeditionstid 9.30--11.30. Telefon 046--222 80 40, e-post \url{expedition@cs.lth.se}

    \item[Hemsida]
    \url{http://cs.lth.se/dod/}. Du bör titta på hemsidan regelbundet! Kort kursinformation finns också i Canvas, men hemsidan är den primära platsen för kursinformation.

    \item[Kursens omfattning]
    \begin{tabular}[t]{@{}ll}
        föreläsningar                & 3 (+1) st \\
        laborationer (obligatoriska) & 3 (+1) st \\
    \end{tabular}

    Från och med i år tillkommer en fjärde föreläsning samt en fjärde laboration som är frivillig. Med start nästa läsår (2025/26) kommer även denna laboration att vara obligatorisk.

    \item[Kurslitteratur]
    All kurslitteratur finns tillgänglig elektroniskt via kurshemsidan. En viss andel av litteraturen delas även ut i pappersform under nollningsveckan, genom institutionens försorg.
    %Kursmaterial delas ut vid den första föreläsningen. I kursmaterialet ingår också kompendiet ''Introduktion till LTH:s Unixdatorer'' som delades ut i samband med datorstugan i introduktionsveckan.
\end{Items}

\clearpage
\section*{Undervisning}

\begin{Items}
    \item[Föreläsningar]
    Tisdagar 15--17 läsvecka 1--4 (kalendervecka 36--39), i föreläsningssal E:A.

    Föreläsningsschema:

    {\hspace{0.5cm}
    \begin{tabular}{lll}
        \emph{Dag}  & \emph{Ämne}                   & \emph{Anmärkning}   \\ \midrule
        Tisdag 3/9  & Introduktion, Linux/Unix      & Förberedelse, lab 1 \\
        Tisdag 10/9 & \LaTeX                        & Förberedelse, lab 2 \\
        Tisdag 17/9 & Versionshantering, Git/GitHub & Förberedelse, lab 3 \\
        Tisdag 24/9 & Maskinkod                     & Förberedelse, lab 4 \\
    \end{tabular}
    }

    \item[Datorlaborationer]
    Onsdagar (15--17), torsdagar (10--12, 13--15 eller 15-17) eller fredagar (10--12 eller 15--17) beroende på grupptillhörighet. Den första laborationen för de första grupperna är onsdag 4/9. Laborationsuppgifterna finns i häftet ''Datorlaborationer, Datorer och datoranvändning''. Observera att laborationerna inleds med en kontrollskrivning som du måste klara för att du ska få delta i laborationen, så du måste förbereda dig inför laborationen genom att läsa igenom uppgiften och göra förberedelserna listade under rubriken ''Hemarbete'' samt se till att du kan svara på frågorna under rubriken ''Kontrollfrågor''.

    Gruppindelningen för laborationerna är densamma som i kursen EDAA45 Programmering, grundkurs. Om du inte läser EDAA45 parallellt, och alltså inte är inplacerad i en grupp, så kontakta kursansvarig via mail (\url{mattias.nordahl@cs.lth.se}) med önskemål om passande laborationstid så kommer du att bli tilldelad en plats i en av grupperna. Vi gör vårt bästa men kan inte fullt ut garantera att du kan laborera den önskade tiden.

    Ämnen för laborationerna:

    {\hspace{0.5cm}
    \begin{tabular}{cl}
        \emph{Laboration} & \emph{Ämne} \\ \midrule
        D1                & Linux/Unix  \\
        D2                & \LaTeX      \\
        D3                & Git/GitHub  \\
        D4                & Maskinkod   \\
    \end{tabular}
    }

    % Tider och salar, läsvecka 1-3:

    % \textit{(se nästa sida)}

    % \newpage

    % {\hspace{0.5cm}
    %     \begin{tabular}{clll}
    %         \emph{Grupp} & \emph{Laborationtillfälle och sal} \\ \midrule
    %         C4           & onsdag 8--10, E:Falk               \\
    %         C5           & onsdag 8--10, E:Hacke              \\
    %         C6           & onsdag 8--10, E:Panter             \\
    %         \\
    %         D11          & onsdag 10--12, E:Falk              \\
    %         D12          & onsdag 10--12, E:Hacke             \\
    %         D13          & onsdag 10--12, E:Panter            \\
    %         D14          & onsdag 10--12, E:Val               \\
    %         \\
    %         D8           & onsdag 13--15, E:Jupiter           \\
    %         D9           & onsdag 13--15, E:Mars              \\
    %         D10          & onsdag 13--15, E:Saturnus          \\
    %         \\
    %         D4           & onsdag 15--17, E:Falk              \\
    %         D5           & onsdag 15--17, E:Hacke             \\
    %         D6           & onsdag 15--17, E:Panter            \\
    %         D7           & onsdag 15--17, E:Val               \\
    %         \\
    %         C1           & torsdag 13--15, E:Elg              \\
    %         C2           & torsdag 13--15, E:Lo               \\
    %         C3           & torsdag 13--15, E:Val              \\
    %         \\
    %         D1           & torsdag 15--17, E:Jupiter          \\
    %         D2           & torsdag 15--17, E:Mars             \\
    %         D3           & torsdag 15--17, E:Saturnus         \\
    %     \end{tabular}
    % }


    Uppsamlingslaborationer kommer att anordnas veckorna efter den ordinarie schemalagda undervisningens slut för att ge alla som missat någon laboration en chans att ta igen dem. Se kurshemsidan för information om när dessa uppsamlingslaborationer arrangeras och hur du gör för att anmäla dig till dem.

\end{Items}

%\newpage

\section*{Examination}


\begin{Items}
    \item[Betyg]
    Endast betygen godkänd/ej godkänd ges. För godkänt kursbetyg så krävs godkänt på de obligatoriska datorlaborationer.
\end{Items}


%\section*{Kursen och COVID-19}
%Årets hösttermin blir en väldigt speciell hösttermin i skuggan av den pågående pandemin då vi måste tänka på att genomföra kursen på ett sådant sätt att risken för smittspridning minimeras. Föreläsningarna kommer därför detta år att ske på distans.
%
%Din medverkan är viktig för att allt ska fungera på ett bra sätt:
%
%\begin{itemize}
%\item Undvik trängsel! Tänk på att hålla avstånd till varandra så gott det går i samband med laborationen. Sprid ut er i datorsalarna så att ni har så stort utrymme mellan varandra som möjligt.
%\item Tvätta händerna ofta. I samband med laborationerna kommer handsprit att finnas tillgänglig i laborationssalarna.
%\item Alla datorer vi använder i kursen har lediga USB-uttag på framsidan av datorlådan. Om du helst föredrar att inte använda de ordinarie tangentbord och datormöss som hör till datorerna kan du ta med ett eget standardtangentbord och mus med USB-anslutning och ansluta dem via dessa uttag. Ordinarie tangentbord/mus behöver inte kopplas ur.
%\item Stanna hemma om du känner de minsta symptom som skulle kunna tyda på en virusinfektion. Anmäl frånvaro enligt instruktionerna i laborationshäftet. Vi kommer under hösten frikostigt arrangera uppsamlingstillfällen som du kan anmäla dig till för att ta igen de laborationer du missar när du stannar hemma så du behöver inte vara rädd för att missa något moment. Se under rubriken \url{Uppsamlingslaborationer} på kurshemsidan för information om när uppsamlingstilfällena går av stapeln och hur du gör för att anmäla dig till dem.
%\end{itemize}

\end{document}