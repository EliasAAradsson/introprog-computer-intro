\documentclass[a4paper,12pt]{article}   % Dokumentklass: artikel, A4, 12 pt
                                        % [a4paper,12pt] är valfria parametrar
                                        % och kan tas bort

% Det finns flera olika LaTeX-kompilatorer. Två vanliga är pdfLaTeX och XeLaTeX.
%  - pdfLaTeX är standard för de flesta dokument och är snabb och pålitlig,
%    men har begränsat Unicode-stöd.
%  - XeLaTeX erbjuder fullt Unicode-stöd och möjligheten att använda
%    systemtypsnitt, vilket gör det lämpligt för internationella dokument.


% Endast för pdfLaTeX. Behövs inte för XeLaTeX, eftersom den använder
% Unicode och systemets redan installerade typsnitt (fonts).
\usepackage[utf8]{inputenc}             % UTF-8 teckenkodning för pdfLaTeX
\usepackage[T1]{fontenc}                % Teckenuppsättning för pdfLaTeX

% Behövs för både för pdfLaTeX och XeLaTeX
\usepackage[swedish]{babel}             % Svensk språkstil och avstavning

% Om du vill använda särskillda typsnitt med XeLaTeX, använd följande:
% \usepackage{fontspec}                  % För att kunna använda typsnitt
% \setmainfont{Latin Modern Roman}       % Sätt huvudtypsnittet

% Övriga paket
\usepackage{graphicx}                   % Hantera bilder
\usepackage{tcolorbox}                  % Färgade ramar/lådor
% \usepackage{amsmath}                    % Mer avancerat matematikstöd,
%                                         % t.ex. flerradiga ekvationer,
%                                         % matriser, anpassade operatorer
%                                         % och bättre layout, men behövs
%                                         % inte för detta dokument

% För att inkludera källkod kan man använda 'verbatim',
% men 'listings' ger ofta bättre resultat
\usepackage{listings}                   % Källkodshantering

% Konfigurera hur kodlistningar ska visas
\lstset{
  language=Scala,                       % Använder Scala för syntaxmarkering
  basicstyle=\ttfamily\footnotesize,    % Använder monospace teckensnitt
                                        % samt mindre textstorlek
  keywordstyle=\color{blue}\bfseries,   % Färglägger nyckelord
  stringstyle=\color{orange},           % Färglägger strängar i orange
  commentstyle=\color{gray}\itshape,    % Färglägger kommentarer
  showstringspaces=false,               % Visar inte mellanslag i strängar
  numbers=left,                         % Visar radnummer till vänster
  numberstyle=\tiny\color{gray},        % Gör radnummer små och grå
  stepnumber=1,                         % Visar varje radnummer
  tabsize=2,                            % Antal mellanslag i en tabb
  breaklines=true,                      % Radbrytning vid behov
  frame=single,                         % Ram runt koden
  columns=fullflexible,                 % Flexibel bredd på bokstäverna 
  keepspaces=true,                      % Behåller mellanslag i koden
  fontadjust=true,                      % Autojusterar fontstorlek vid behov
}


\usepackage{enumitem} % Anpassade listor
% Anpassa hur 'description' listor ser ut
\setlist[description]{
    leftmargin=2.3cm,   % Vänstermarginal
    rightmargin=1cm,    % Högermarginal
    labelindent=1cm,    % Etikettindrag
    labelwidth=1cm,     % Etikettbredd
    align=right,        % Högerjustera etikett
    labelsep=3mm        % Avstånd mellan etikett och beskrivning
}


\title{What is the meaning of life?}
\author{Unnamed Author}
\date{\today}

\begin{document}

\maketitle              % Skriv ut titel, författare och datum

\[ 42 \]

Resten av dokumentet här!

% Du kommer behöva söka upp information om flera kommandon. Ta hjälp av
% kursmaterialet och internet. Nedan följer lite tips:
%
%  - För att forcera en ny sida, använd \newpage
%  - Kolla up skillnaden mellan \emph, \textit och \textbf
%  - För att inkludera en figur, använd figure och includegraphics
%  - För att skriva kod kan man använda verbatim och verbatiminput, men...
%  - ... det blir snyggare med listings-paketet
%  - LaTeX numrerar automatiskt rubriker och figurer
%  - Bilagor numreras med bokstäver, och heter "Appendix" på engleska
%  - Med \ref kan man referera till saker som är märkta med en \label,
%    t.ex. figurer, tabeller, ekvationer och sektioner
%  - Fina lådor kan skapas med:
%      \begin{tcolorbox}[title=Min titel här]
%        Text...
%      \end{tcolorbox}
%  - Det finns tre huvudtyper av listor: itemize, enumerate och description
%  - Med tecknet ~ skriver man ett hårt mellanslag, som inte bryts över
%    radslut. Användbart t.ex. för att skriva "figur 1" utan att ettan
%    hamnar på nästa rad.
%  - Figurer och tabeller kan placeras med [h], [t], [b] och [p] för att
%    tvinga LaTeX att placera dem 'here', 'top', 'bottom' ner eller på en
%    egen sida, 'page'. \begin{table}[b] ...
%  - Skriv figurtexter med \caption
%  - Skriv fotnötter med \footnote{Text här}, där du vill ha fotnoten.
%    Fotnotstexten hamnar automatiskt längst ner på sidan.


\end{document}
