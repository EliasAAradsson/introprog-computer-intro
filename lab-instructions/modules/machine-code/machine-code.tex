
\section{Laboration \arabic{section} --- Maskinkod}

\emph{Mål:} Att förstå hur en dator fungerar på en grundläggande nivå. Man hör ibland att datorer endast förstår och arbetar med ``ettor och nollor'', men vad betyder det egentligen? I denna laboration kommer vi arbeta med en förenklad datormodell där vi kan inspektera och manipulera minnet och processorn på en låg nivå och se just hur ettor och nollor används för att utföra beräkningar.

\subsection{Kort teori}
% ta upp:
%  - CPU, vad gör den? Vad är register?
%  - (Arbets)Minne. Lådor där man kan stoppa in värden, addresseras med nummer.
%  - Instruktioner. En CPU har en uppsättning instruktioner den kan utföra.
%  - Maskinkod. Instruktioner representeras som binära tal.
%  - Teckenkodning. Hur representeras bokstäver och andra tecken i minnet?
%    - ASCII, enkel tabell som mappar binärt värde till symbol.
%    - Unicode, utökad tabell som stödjer fler språk och symboler. Används inte i labben.
%  - Programräknare. Ett speciellt register som håller koll på vilken instruktion som ska utföras härnäst.



\subsection*{Hemarbete}
\begin{Hemarbete}\firmlist
    \item Titta igenom föreläsningsbilderna från föreläsning 4 (titel todo).
    \item Läs igenom ...
    \item Läs igenom uppgifterna under rubriken Datorarbete.
    \item Förbered din dator för laborationen genom att installera Java om du inte redan gjort det. Instruktioner finns på kurshemsidan.
    \item Ladda ned programmet \textbf{\progname{}} från kurshemsidan och packa upp det i en lämplig katalog. Programmet består av en \texttt{.jar}-fil som innehåller allt som behövs för att köra programmet.
    \item Kontrollera att du kan köra programmet genom att dubbelklicka på \texttt{.jar}-filen. Om det inte fungerar kan du istället starta programmet från terminalen med kommandot:
    \begin{center}
        {\code{java -jar \progfilename{}}}
    \end{center}
    \item Läs igenom användarmanualen för programmet som beskriver hur det fungerar. Manualen öppnas automatiskt när programmet startas, eller via menyn \texttt{Help}~$\rightarrow$~\texttt{Show Help}.
\end{Hemarbete}
Länkar till det ovan refererade materialet finns på kurshemsidan under  ''Datorlaborationer'' om du inte har pappersversionerna av dem.

\subsection*{Kontrollfrågor}
\begin{Kontrollfragor}
    \item Todo.
\end{Kontrollfragor}

\subsection*{Bakgrund}
Här följer en kort sammanfattning av materialet ni har läst som hemarbete, som kommer vara användbart under laborationen.

% 

\clearpage
\subsection*{Datorarbete}
Under laborationen kommer vi att arbeta med programmet \progname{} som simulerar en förenklad dator. Programmet är skrivet i Java och kan köras på de flesta datorer. Datorn har ett minne och en processor med register som kan utföra enkla operationer. Programmet låter oss inspektera och manipulera minnet och processorn på en låg nivå och se hur datorn tolkar och utför instruktioner.

\begin{Datorarbete}
    \item Börja med att starta \progname{}. Om du inte gjort det redan så läs igenom användarmanualen som beskriver hur programmet fungerar. Den öppnas automatiskt när programmet startas, eller via menyn \texttt{Help}~$\rightarrow$~\texttt{Show Help}.

    \item I programmet finns ett antal exempelprogram som kan laddas in, under menyn \texttt{Examples}. Öppna det exempel som heter \texttt{Tiny program}. Detta är ett mycket enkelt program som bara använder fyra minnesceller, varav tre innehåller instruktioner och en innehåller data. Prova att stega igenom programmet och se vad som händer vid varje steg. Nedan är en beskrivning av vad programmet gör:
    \begin{enumerate}
        \item Eftersom programräknaren startar på värdet 0 kommer processorn att läsa värdet på den minnesplatsen först och tolka det som en instruktion.
        \item Instruktionen är \texttt{LOD (dst: OUT)}, eller "Load into OUT", vilket betyder att värdet i nästa minnescell ska laddas in till registret som heter \texttt{OUT} (registret med index 6).
        \item När instruktionen körs så läses alltså värdet på minnesplats 1, och kopieras till registret \texttt{OUT}. Därefter ökar programräknaren med 2, eftersom instruktionen är två byte lång.
        \item Nu pekar programräknaren på minnesplats 2, som innehåller instruktionen \texttt{PRT}, eller "Print text". Denna instruktion skriver ut värdet som finns i registret \texttt{OUT} till konsolen, som ett ASCII-tecken. Därefter ökar programräknaren med 1.
        \item Till sist så körs instruktionen på minnesplats 3, som är \texttt{HLT}, eller "Halt", som avslutar programmet.
    \end{enumerate}

    \item Bygg vidare på det tidigare exemplet och skriv ut flera tecken i rad, t.ex. \texttt{Hello,~World!}. Kom ihåg att varje tecken måste representeras som ett ASCII-värde. Ta hjälp av ASCII-tabellen för att hitta tecken och deras motsvarande decimal- eller binärvärden.

    \noindent När du är klar så spara ditt program genom att klicka på \texttt{File}~$\rightarrow$~\texttt{Save} och ge filen ett namn. Du kan sedan ladda in programmet igen genom att klicka på \texttt{File}~$\rightarrow$~\texttt{Open} och välja filen.
    \hint{Tecknet " " (mellanslag) har ASCII-värdet 32.}

    \item Från \texttt{Examples}-menyn, öppna nu det exempel som heter \texttt{Simple addition}, och försök förstå vad det gör och hur det fungerar. Kom ihåg och ta hjälp av att varje minnescell visas med olika tolkningar, däribland decimalt och som instruktioner. Vilka värden kommer att tolkas som instruktioner och vilka som data? Vad kommer resultatet att bli när programmet körs? Till sists, kör programmet och se om du hade rätt.

    \item Modifiera programmet så att det istället utför subtraktion, samt läser de två operanderna från specifika minnesplatser, säg 18 och 19 istället. Ändra värdena på minnesplatserna 18 och 19 för att utföra beräkningen \(127-43\) och kör programmet för att se att det fungerar.

    Vad tror du händer om man vänder på operanderna, dvs. utför beräkningen \(43-127\)? Prova att byta plats på minnesplatserna 18 och 19 och kör programmet för att se om du hade rätt. När du är klar så spara ditt program i en ny fil \texttt{subtraction.txt}.
    \hint{Använd instruktionen \texttt{LDA}, alltså "Load Address".}

    \item Skriv ett eget program som innehåller en \texttt{if}-sats. Programmet ska jämföra två tal och skriva ut Y (för "yes") om de är lika, annars N (för "no"). Använd hoppinstruktioner för att implementera villkorlig exekvering. Kör programmet och se att det fungerar som förväntat.

    \item Öppna exempelprogrammet \texttt{Simple loop} och försök förstå vad det gör och hur det fungerar. Vad kommer resultatet att bli när programmet körs? Kör programmet och se om du hade rätt.

    \item Vi kan alltså skapa loopar genom att använda hoppinstruktioner. Skapa ett eget program som räknar från 0 till 10 och skriver ut varje tal. Använd en loop för att uppnå detta. Kör programmet och se att det fungerar som förväntat.
\end{Datorarbete}


\subsection{Extrauppgifter (frivilliga)}

\begin{Datorarbete}
    \item Skriv ett program som konverterar en följd av siffror till en sträng av motsvarande tecken. Programmet ska läsa in en följd av värden (som måste vara 0--9) från minnet och konvertera dem till motsvarande ASCII tecken. Alltså, den första följden av siffror ska kunna skrivas till terminalen med \texttt{PRD} instruktionen (print decimal), medan de konverterade värdena ska kunna skrivas ut med \texttt{PRT} instruktionen (print text).

    \item Skriv ett program som gör \emph{två} jämförelser i ett "och"-uttryck. Programmet ska kontrollera ifall två tal båda är större än 50, och i så fall skriva ut "True", annars "False".

    \item Implementera ett funktionsanrop i ditt program. Skapa en funktion som adderar två tal och returnerar summan. Använd hoppinstruktioner för att anropa funktionen och hantera returvärdet. Ditt program ska motsvara följande pseudokod:
    \begin{verbatim}
		1. a = 3
		2. b = 4
		4. sum = call add(a, b)
		5. print sum
		6. halt
		7. add(a, b):
		8.   return a + b
	\end{verbatim}
    \hint{Se nedan för förslag på hur en funktion kan implementeras. Notera att termen "funktion" används ganska löst här, eftersom det inte finns något koncept av funktioner i maskinkod.}
    \begin{enumerate}
        \item Välj för din funktion fyra register som ska ha särskild betydelse:
              \begin{itemize}
                  \item Ett register för att spara det första talet (första funktionsparametern).
                  \item Ett register för att spara det andra talet (andra funktionsparametern).
                  \item Ett register för att spara programräknarens nuvarande värde, så att funktionen vet vart den ska hoppa tillbaka.
                  \item Ett register för att spara returvärdet, så att koden som anropade funktionen kan använda det.
              \end{itemize}

              Använd förslagsvis registerna \texttt{R1}--\texttt{R3} för att spara de två talen och programräknarens värde. Returvärdet behövs inte förrän efter beräkningen är klar, så du kan återanvända ett av de två talregistren för att spara det, eller använda \texttt{RES}.
        \item För koden som ska anropa funktionen:
              \begin{enumerate}
                  \item Spara de två talen och programräknarens nuvarande värde i dina valda register.
                  \item Hoppa till funktionens startadress.
              \end{enumerate}
        \item För funktionen:
              \begin{enumerate}
                  \item Använd de två talregistren för att hämta de två talen och utför beräkningen.
                  \item Spara resultatet i returregistret.
                  \item Använd det sparade programräknarvärdet för att hoppa tillbaka till den instruktion som anropade funktionen.
              \end{enumerate}
        \item Tänk på att funktionen ska retunera och fortsätta med instruktionen \emph{efter} anropet, så att programmet inte hamnar i en loop.

    \end{enumerate}

    \item Skriv ett program som multiplicerar två tal genom upprepade additioner. Använd en loop för att utföra multiplikationen och spara resultatet i en minnescell.

    \item När en funktion anropas är det viktigt att inte överskriva nuvarande värden i vissa register som används av programmet. I verkliga processorer finns det därför speciella register kända som \emph{mottagarbevarade} (eng. \emph{callee-saved}), vilka ska sparas och återställas av den anropade funktionen. Implementera en funktion som använder och tillfälligt sparar registren \texttt{R1} till \texttt{R3} och återställer dem innan funktionen avslutas.

    \hint{Använd ett ledigt område i minnet (i verkliga system skulle detta hanteras av operativsystemet) för att temporärt spara de register som ska bevaras. Nu kan funktionen fritt använda registerna till sina beräkningar. Innan funktionen avslutas och returnerar, återställ dessa register till deras ursprungliga värden.}
\end{Datorarbete}

% Brainstorming: Vilka uppgifter ska ges till studenterna?
% - Skriv ett program som räknar ut summan av alla tal i en lista.
% - Skriv ut en enkel beräkning som text, t.ex. "2 + 3 = 5".
% - Implementera en if-sats med hjälp av hoppinstriktioner, t.ex. skriv ut det största av två tal. Utöka programmet så att det kan jämföra tre tal.
% - Implementera en loop som räknar upp ett tal från 0 till 10 och skriver ut det.
% - Implementera en loop som summerar ett godtyckligt antal tal i minnet.
% - Använd "load address" instruktionen, t.ex. för att hitta ett givet värde i minnet och skriva ut det samt dess adress.
% - Implementera en funktion, alltså en del av programmet som kan anropas från andra delar av programmet. Det kommer innefatta:
%   - Spara programräknarens nuvarande värde.
%   - Hoppa till funktionen.
%   - Utföra funktionens instruktioner. T.ex. addera två tal och returnera resultatet.
%   - Hur ska funktionen returnera resultatet? Lägg det i ett särskilt register, eller skriv det till en viss adress i minnet?
%   - Hoppa tillbaka till det sparade programräknarvärdet.
% - Implementera ett program som leder till "segmentation fault", alltså editerar minnet utanför dess tillåtna område och t.ex. skriver över instruktioner.



% Nedan förslag är genererade av Chat-GPT


% Grundläggande Uppgifter

%     Läs och förstå minnet: Skriv ett program som läser ett
%     specifikt värde från en definierad adress i minnet och skriver ut det på
%     skärmen.

%     Enkel addition: Implementera ett program som läser två
%     tal från specifika minnesplatser, adderar dem och skriver resultatet till
%     en annan plats i minnet.

%     Decimal till binär omvandling: Skriv ett program som tar
%     ett decimaltal och konverterar det till binär form, sedan lagrar
%     binärtalen i minnet.


% Mellannivå Uppgifter

%     Implementera en loop: Skapa ett program som använder en
%     loop för att addera en sekvens av tal från minnet och skriver ut den
%     totala summan.

%     Villkorslogik: Skriv ett program som jämför två tal och
%     skriver ut vilket som är större, eller om de är lika.

%     Implementera en enkel räknare: Skapa ett program som
%     räknar från 1 upp till ett nummer som anges av användaren och visar varje
%     steg i konsolen.


% Avancerade Uppgifter

%     Multiplicera utan *-operatorn: Skriv ett
%     program som utför multiplikation av två tal genom upprepade additioner.

%     Använda funktionsanrop: Implementera en subrutin som
%     utför en matematisk operation (t.ex. addition) och kan återanvändas av
%     huvudprogrammet.

%     Skapa en miniräknare: Implementera en enkel miniräknare
%     som kan utföra grundläggande aritmetik (addition, subtraktion,
%     multiplikation, division) baserat på användarinmatning.


% Frivilliga Uppgifter

%     Segmentation Fault Simulator: Implementera ett program
%     som medvetet försöker åtkomst utanför tilldelat minnesområde för att
%     simulera ett segmentation fault och diskutera dess konsekvenser.

%     Optimering: Skriv ett program och sedan försök att
%     optimera det för att köra snabbare eller använda mindre minne, och
%     dokumentera skillnaderna.

%     Algoritmisk komplexitet: Implementera två olika
%     algoritmer för att lösa samma problem och jämför deras prestanda i termer
%     av exekveringstid och minnesanvändning.


% Diskussionsfrågor och Reflektioner

%     Lågnivå vs. Högnivå Programmering:

%         Hur skiljer sig upplevelsen av att programmera i en lågnivåmiljö
%         jämfört med högnivåspråk som Java eller Python?

%         Vilka är fördelarna och nackdelarna med att arbeta närmare hårdvaran?

%     Felsökning:

%         Vilka strategier är mest effektiva när du felsöker lågnivåprogram?

%         Hur kan verktyg och visualiseringar förbättra förståelsen för vad ditt
%         program gör?

%     Datorarkitektur:

%         Hur har denna laboration påverkat din förståelse för hur datorer
%         fungerar?

%         Diskutera vikten av operativsystemet och dess roll i att hantera
%         hårdvaruresurser.



% Nedan är också genererat av Chat-GPT

\subsection*{Fortsatta uppgifter}
\begin{Datorarbete}
    \item Förstå och implementera kontrollflöden:
    \begin{enumerate}
        \item Skapa ett program som använder en enkel \texttt{if}-sats för att bestämma och skriva ut det största av två inmatade tal. Använd hoppinstruktioner för att utföra villkorliga hopp.
        \item Utöka programmet för att jämföra tre tal och skriva ut det största.
    \end{enumerate}

    \item Implementera loopar:
    \begin{enumerate}
        \item Skriv ett program som räknar från 0 till 10 och skriver ut varje tal.
        \item Skriv ett program som räknar ut summan av alla tal i en lista lagrade i minnet. Använd en loop för att gå igenom listan.
    \end{enumerate}

    \item Använda avancerade instruktioner:
    \begin{enumerate}
        \item Skriv ett program som använder \texttt{load address} instruktionen för att hitta ett specifikt värde i minnet och skriva ut värdet samt dess adress.
    \end{enumerate}

    \item Funktioner och subrutiner:
    \begin{enumerate}
        \item Skapa en funktion som adderar två tal och returnerar resultatet. Implementera funktionen så att den kan anropas från andra delar av programmet, inklusive:
              \begin{itemize}
                  \item Spara programräknarens nuvarande värde.
                  \item Hoppa till funktionens startadress.
                  \item Utföra additionen och lagra resultatet på ett bestämt sätt (i ett register eller på en specifik minnesadress).
                  \item Återgå till programräknarens sparade värde efter att funktionen har körts.
              \end{itemize}
    \end{enumerate}

    \item Felsökning och felhantering:
    \begin{enumerate}
        \item Implementera ett program som medvetet leder till en \texttt{segmentation fault} genom att försöka skriva till eller läsa från en minnesadress utanför tillåtet område. Diskutera vad som händer och varför det är viktigt att hantera sådana fel.
    \end{enumerate}
\end{Datorarbete}

\subsection*{Sammanfattning och reflektion}
Under laborationen har ni fått utforska grunderna i hur datorns processor och minne samarbetar för att utföra instruktioner och hantera data. Diskutera i grupp:
\begin{itemize}
    \item Vilka var de största utmaningarna med att programmera på denna låga nivå?
    \item Hur skiljer sig det från att skriva program i högnivåspråk som Java eller Python?
    \item Vilken insikt ger detta om betydelsen av operativsystem och abstraktioner i mjukvaruutveckling?
\end{itemize}

Innan laborationen avslutas, se till att spara era program och dela dem med kursledaren. Diskutera eventuella frågor eller problem som uppstått under laborationen med er handledare.



\newpage

\section{Java -- Felsökning}
\label{sec:java_troubleshoot}

Om du inte kan köra \progname programmet kan det beror på att Java inte är korrekt installerat på din dator. Följande steg beskriver hur du kan felsöka och lösa problemet.

\begin{itemize}
    \item \textbf{Kontrollera att Java är installerat:} Öppna en terminal och skriv \texttt{java -version}. Om du får ett felmeddelande betyder det att Java inte är installerat. Installera Java genom att följa instruktionerna på kurshemsidan.
    \item ...
\end{itemize}