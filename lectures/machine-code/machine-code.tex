
\title{Maskinkod}
\section{Maskinkod}


\begin{frame}
    \begin{block}{\centering\Large Föreläsning \arabic{section} --- Maskinkod}
        Förberedelse inför laboration 4.

        \begin{itemize}
            \item Vad är en dator?
            \item Binära tal.
            \item CPU och Minne.
            \item Maskinkod och instruktioner.
            \item Assembly-språk.
            \item Kompilerare och tolkar.
            \item c3pu (labförberedelse).
        \end{itemize}
    \end{block}
\end{frame}

\begin{frame}[fragile=singleslide]
    \frametitle{Bakgrund}
    \begin{columns}
        \begin{column}{0.75\textwidth}
            \begin{block}{Varför lära sig Maskinkod?}
                \begin{itemize}
                    \item \textbf{Effektiv Problemlösning:} Förstå hur datorer utför uppgifter kan leda till snabbare och mer effektiva lösningar när du programmerar.
                    \pause
                    \item \textbf{Förbättrade felsökningsfärdigheter:} Kunskap om maskinkod kan avslöja djupliggande orsaker till buggar, vilket gör det enklare att fixa komplexa problem.
                    \pause
                    \item \textbf{Bättre teknikval:} Grundläggande förståelse för maskinkod kan vägleda dig i val av hårdvara och optimeringar, likt valet av motor för en bil baserat på användningsområde.
                    \pause
                    \item \textbf{Innovativt tänkande:} Förståelsen för datorns fundamentala operationer möjliggör kreativa och nyskapande tekniklösningar.
                \end{itemize}
            \end{block}
        \end{column}
        \begin{column}{0.25\textwidth}
            \begin{center}
                \begin{tikzpicture}
                    \node (hl) [rectangle, draw, fill=blue!20, text width=6em, text centered, minimum height=2em] {Högnivåspråk};
                    \node (al) [rectangle, draw, fill=green!20, below=of hl, text width=6em, text centered, minimum height=2em] {Assembly};
                    \node (mc) [rectangle, draw, fill=red!20, below=of al, text width=6em, text centered, minimum height=2em] {Maskinkod};
                    \node (hw) [rectangle, draw, fill=orange!20, below=of mc, text width=6em, text centered, minimum height=2em] {Hårdvara};

                    \draw[->] (hl) -- (al);
                    \draw[->] (al) -- (mc);
                    \draw[->] (mc) -- (hw);
                \end{tikzpicture}
            \end{center}
        \end{column}
    \end{columns}
\end{frame}


\begin{frame}[fragile=singleslide]
    \frametitle{Vad är en dator?}
    \begin{columns}[T] % The [T] option is used to align the columns content at the top
        \begin{column}{0.6\textwidth}
            \begin{block}{Komponenter}
                \begin{itemize}
                    \item \textbf{CPU:} Datorns hjärna, innehåller ALU, kontrollenhet, och internminne (cache).
                    \pause
                    \item \textbf{Minne:} Lagring av data och instruktioner. RAM och permanent lagring (t.ex., hårddisk).
                    \pause
                    \item \textbf{I/O-enheter:} För inmatning och utmatning, t.ex., mus, tangentbord, skärm.
                \end{itemize}
            \end{block}
            \begin{block}{Programvara vs. Hårdvara}
                \begin{itemize}
                    \item \textbf{Hårdvara:} Fysiska delarna av datorn.
                    \item \textbf{Programvara:} Instruktioner för att utföra uppgifter.
                \end{itemize}
            \end{block}
        \end{column}
        \begin{column}{0.4\textwidth}
            \begin{center}
                \begin{tikzpicture}[scale=0.8, every node/.style={scale=0.8}]
                    % Define CPU and its internal components
                    \node (cpu) [rectangle, draw, fill=blue!20, minimum height=9.2em, minimum width=9.5em, rounded corners, align=center, label=above:CPU] {};
                    \node (cu) [rectangle, draw, fill=green!30, below=.5em of cpu.north, minimum height=1.5em, minimum width=9em, align=center] {Control Unit};
                    \node (alu) [rectangle, draw, fill=red!30, below=.5em of cu, minimum height=1.5em, minimum width=9em, align=center] {ALU};
                    \node (pm) [rectangle, draw, fill=orange!40, below=.5em of alu, minimum height=1.5em, minimum width=9em, align=center] {Primary Cache};
                    \node (sm) [rectangle, draw, fill=yellow!40, below=.2em of pm, minimum height=1.5em, minimum width=9em, align=center] {Secondary Cache};
                    
                    % Define I/O Devices
                    \node (io) [rectangle, draw, fill=gray!30, below=of cpu, minimum height=3em, minimum width=9.5em, rounded corners, align=center, label=above:I/O Devices] {Mouse, Keyboard,\\Screen, etc.};

                    % Define RAM
                    \node (ram) [rectangle, draw, fill=purple!40, left=of cpu, yshift=-3.1em, xshift=3em, minimum height=15.5em, minimum width=3em, rounded corners, align=center, label=above:RAM] {};

                \end{tikzpicture}
            \end{center}
        \end{column}
    \end{columns}
    
\end{frame}

\begin{frame}[fragile=singleslide]
    \frametitle{Binära tal - Introduktion}
    \begin{block}{Olika talsystem}
        \begin{itemize}
            \item Talsystem används för att representera tal med symboler.
            \item Decimal (bas 10) - vanligaste, använder siffror 0-9.
        \end{itemize}
    \end{block}
    \begin{block}{Binärt talsystem}
        \begin{itemize}
            \item Binärt (bas 2) - används av datorer, använder bara 0 och 1.
            \item Varje siffra kallas en \textit{bit}.
        \end{itemize}
    \end{block}
    \begin{block}{Varför binära tal i datorer?}
        \begin{itemize}
            \item Enkelt att representera elektroniskt (ström finns/ström saknas).
            \item Pålitligt att tolka signaler som hög (1) eller låg (0) spänning.
        \end{itemize}
    \end{block}
\end{frame}

\begin{frame}[fragile=singleslide]
    \frametitle{Binära tal - I Datorer}
    \begin{block}{Bits, Bytes och Ord}
        \begin{itemize}
            \item \textbf{Bit:} Grundläggande enhet av data i datorer (0 eller 1).
            \item \textbf{Byte:} Grupp av 8 bitar.
            \item \textbf{Ord:} Processor-specifik grupp av bitar (t.ex., 32-bit eller 64-bit).
        \end{itemize}
    \end{block}
    \begin{block}{Användning av binära tal}
        \begin{itemize}
            \item Lagring och bearbetning av all data och instruktioner.
            \item Varje byte kan representera ett tecken i text, t.ex., ASCII-kod.
            \item Större grupper (ord) hanterar mer komplexa data som heltal och flyttal.
        \end{itemize}
    \end{block}
\end{frame}

\begin{frame}[fragile=singleslide]
    \frametitle{Hexadecimala Tal}
    \begin{block}{Hexadecimala Talsystemet}
        \begin{itemize}
            \item Hexadecimal (bas 16) - använder siffror 0-9 och bokstäverna A-F.
            \item Varje tecken representerar fyra binära siffror (bitar).
        \end{itemize}
    \end{block}
    \begin{block}{Varför Hexadecimala Tal?}
        \begin{itemize}
            \item \textbf{Förenkling:} Lättare att läsa och skriva stora binära tal.
            \item \textbf{Minnesadressering:} Används ofta för att representera minnesadresser. Hjälpsamt för att hitta och felsöka minnesproblem.
        \end{itemize}
    \end{block}
    \begin{block}{Exempel}
        \begin{itemize}
            \item Binärt: \texttt{1011 1010} = Hex: \texttt{BA}
            \item Större binärtal: \texttt{1011 1010 0101 1110} = Hex: \texttt{BA5E}
        \end{itemize}
    \end{block}
\end{frame}




\begin{frame}[fragile=singleslide]
    \frametitle{Central Processing Unit (CPU)}
    % Structure and function of the CPU
    %  - ALU, registers, control unit
    %  - Role of the CPU in executing machine code
\end{frame}

\begin{frame}[fragile=singleslide]
    \frametitle{Minne i datorer}
    % Types of memory: RAM, ROM, cache
    % How memory works with the CPU
\end{frame}

\begin{frame}[fragile=singleslide]
    \frametitle{Grundläggande om maskinkod}
    % Definition and characteristics of machine code
    % Differences between machine code, assembly language, and high-level programming languages
\end{frame}

\begin{frame}[fragile=singleslide]
    \frametitle{Maskininstruktioner}
    % Components of a machine instruction: opcode, operands
    % How instructions are executed by the CPU
\end{frame}

\begin{frame}[fragile=singleslide]
    \frametitle{Assembly-språk}
    % Introduction to assembly language as a low-level programming language
    % Relationship between assembly language and machine code
\end{frame}

\begin{frame}[fragile=singleslide]
    \frametitle{Skriva enkel maskinkod}
    % Example of a simple program in machine code
    % Explanation of what each part of the program does
\end{frame}

\begin{frame}[fragile=singleslide]
    \frametitle{Maskinkod och moderna datorer}
    % Relevance of machine code in contemporary software development
    % Overview of how high-level languages are translated into machine code
\end{frame}

\begin{frame}[fragile=singleslide]
    \frametitle{Kompilatorer och tolkar}
    % Explanation of compilers and interpreters
    % How they translate high-level code into executable machine code
\end{frame}

\begin{frame}[fragile=singleslide]
    \frametitle{Utmaningar med att arbeta med maskinkod}
    % Difficulties and limitations of coding in machine code
    % Importance of understanding machine code for debugging and optimization
\end{frame}

\begin{frame}[fragile=singleslide]
    \frametitle{Sammanfattning}
    % Recap of the key points covered
    % Encouragement to explore more about how programming languages work at the machine level
\end{frame}

\begin{frame}[fragile=singleslide]
    \frametitle{Frågor och diskussion}
    % Open the floor for questions
    % Suggest further reading materials and resources
\end{frame}