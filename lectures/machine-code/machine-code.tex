
\title{Maskinkod}
\section{Maskinkod}


\begin{frame}[fragile,t]
    \begin{block}{\centering\Large Föreläsning 4 --- Maskinkod}

        \halfblankline

        \ti{Förberedelse inför laboration 4}

        \halfblankline
        \begin{itemize}
            \ii{Vad är en dator?}
            \ii{Binära tal}
            \ii{CPU och Minne}
            \ii{Maskinkod och instruktioner}
            \ii{Assembly-språk}
            \ii{Kompilerare och tolkar}
            \ii{\emph{c3pu} -- används på laborationen}
        \end{itemize}

        \halfblankline
    \end{block}

\end{frame}

\begin{frame}[fragile,t]
    \frametitle{Bakgrund}
    \vspace{2em}

    \begin{columns}[T]
        \begin{column}{0.65\textwidth}
            \ti{Varför lära sig Maskinkod?}

            \blankline
            \begin{itemize}
                \ii{Djupare förståelse -- insikt i hur datorer fungerar på en grundläggande nivå.}
                \begin{itemize}
                    \ii{\emph{Vad händer i datorn med kod som vi skrivit?}}
                \end{itemize}
                \ii{Effektiv problemlösning och innovativt tänkande.}
                \begin{itemize}
                    \ii{\emph{Hur kan vi skriva snabb och effektiv kod?}}
                    \ii{\emph{Hur kan vi hitta på innovativa lösningar?}}
                \end{itemize}
            \end{itemize}
        \end{column}
        \begin{column}{0.35\textwidth}
            \vspace{-2em}
            \begin{center}
                \begin{tikzpicture}
                    \only<+->{
                        \node (hl) [rectangle, draw, fill=blue!20, text width=6em, text centered, minimum height=2em] {Högnivåspråk};
                        \node (al) [rectangle, draw, fill=green!20, below=of hl, text width=6em, text centered, minimum height=2em] {Assembly};
                        \node (mc) [rectangle, draw, fill=red!20, below=of al, text width=6em, text centered, minimum height=2em] {Maskinkod};
                        \node (hw) [rectangle, draw, fill=orange!20, below=of mc, text width=6em, text centered, minimum height=2em] {Hårdvara};

                        \draw[->] (hl) -- (al);
                        \draw[->] (al) -- (mc);
                        \draw[->] (mc) -- (hw);
                    }
                \end{tikzpicture}
            \end{center}
        \end{column}
    \end{columns}
\end{frame}


\begin{frame}[fragile,t]
    \frametitle{Vad är en dator?}


    \begin{columns}[T] % The [T] option is used to align the columns content at the top
        \begin{column}{0.6\textwidth}
            \ti{Komponenter}

            \halfblankline
            \begin{itemize}
                \ii{\textbf{CPU:} ALU, kontrollenhet, cache}
                \ii{\textbf{Minne:} RAM och lagring (hårddisk)}
                \ii{\textbf{I/O-enheter:} In- och utmatning (mus, tangentbord, skärm)}
                \ii{\textbf{Moderkort:} Kopplar ihop alla komponenter}
            \end{itemize}

            \blankline
            \ti{Programvara vs. Hårdvara}

            \halfblankline
            \begin{itemize}
                \ii{\textbf{Hårdvara:} Fysiska delarna av datorn}
                \ii{\textbf{Mjukvara (programvara):} Instruktioner för att utföra uppgifter}
            \end{itemize}

        \end{column}
        \begin{column}{0.4\textwidth}
            \begin{center}
                \begin{tikzpicture}[scale=0.8, every node/.style={scale=0.8}]
                    % Define CPU and its internal components
                    \node (cpu) [rectangle, draw, fill=blue!20, minimum height=9.2em, minimum width=9.5em, rounded corners, align=center, label=above:CPU] {};
                    \node (cu) [rectangle, draw, fill=green!30, below=.5em of cpu.north, minimum height=1.5em, minimum width=9em, align=center] {Control Unit};
                    \node (alu) [rectangle, draw, fill=red!30, below=.5em of cu, minimum height=1.5em, minimum width=9em, align=center] {ALU};
                    \node (pm) [rectangle, draw, fill=orange!40, below=.5em of alu, minimum height=1.5em, minimum width=9em, align=center] {Primary Cache};
                    \node (sm) [rectangle, draw, fill=yellow!40, below=.2em of pm, minimum height=1.5em, minimum width=9em, align=center] {Secondary Cache};

                    % Define I/O Devices
                    \node (io) [rectangle, draw, fill=gray!30, below=of cpu, minimum height=3em, minimum width=9.5em, rounded corners, align=center, label=above:I/O Devices] {Mouse, Keyboard,\\Screen, etc.};

                    % Define RAM
                    \node (ram) [rectangle, draw, fill=purple!40, left=of cpu, yshift=-3.1em, xshift=3em, minimum height=15.5em, minimum width=3em, rounded corners, align=center, label=above:RAM] {};

                \end{tikzpicture}
            \end{center}
        \end{column}
    \end{columns}

\end{frame}

\begin{frame}[fragile,t]
    \frametitle{Binära tal - Introduktion}


    \ti{Olika talsystem}

    \halfblankline
    \begin{itemize}
        \ii{Talsystem används för att representera tal med symboler.}
        \ii{Decimal (bas 10) -- vanligast, använder siffror 0-9.}
    \end{itemize}

    \blankline
    \ti{Binärt talsystem}

    \halfblankline
    \begin{itemize}
        \ii{Binärt (bas 2) -- används av datorer, bara två siffror, 0 och 1.}
        \ii{Varje siffra kallas en \textit{bit}.}
    \end{itemize}

    \blankline
    \ti{Varför binära tal i datorer?}

    \halfblankline
    \begin{itemize}
        \ii{Enkelt att representera elektroniskt (ström finns/ström saknas).}
        \ii{Pålitligt att tolka signaler som hög (1) eller låg (0) spänning.}
    \end{itemize}

\end{frame}

\begin{frame}[fragile,t]
    \frametitle{Binära tal - I Datorer}


    \ti{Bits, Bytes och Ord}

    \halfblankline
    \begin{itemize}
        \ii{\textbf{Bit:} Grundläggande enhet av data i datorer (0 eller 1).}
        \ii{\textbf{Byte:} Grupp av 8 bitar.}
        \ii{\textbf{Ord:} Processor-specifik grupp av bitar (t.ex., 32-bit eller 64-bit).}
    \end{itemize}

    \blankline
    \ti{Användning av binära tal}

    \halfblankline
    \begin{itemize}
        \ii{Lagring och bearbetning av all data och instruktioner.}
        \ii{Varje byte kan representera ett tecken i text, t.ex., ASCII-kod.}
        \ii{Större grupper (ord) hanterar mer komplexa data som heltal och flyttal.}
    \end{itemize}

\end{frame}

\begin{frame}[fragile,t]
    \frametitle{Hexadecimala Tal}

    \ti{Hexadecimala Talsystemet}

    \begin{itemize}
        \ii{Hexadecimal (bas 16) - använder siffror 0-9 och bokstäverna A-F.}
    \end{itemize}

    \blankline
    \ti{Varför Hexadecimala Tal?}

    \begin{itemize}
        \ii{Förenkling!}
        \begin{itemize}
            \ii{Varje tecken representerar fyra binära siffror (bitar).}
            \ii{Lättare att läsa och skriva stora binära tal.}
        \end{itemize}
        \ii{Används ofta för att representera \emph{minnesadresser}}
        \ii{Hjälpsamt för att hitta och felsöka minnesproblem.}
    \end{itemize}

    \blankline
    \ti{Exempel}

    \begin{itemize}
        \ii{Binärt: \texttt{1011 1010} = Hex: \texttt{BA}}
        \ii{Större binärtal: \texttt{1011 1010 0101 1110} = Hex: \texttt{BA5E}}
    \end{itemize}

\end{frame}




\begin{frame}[fragile,t]
    \frametitle{Central Processing Unit (CPU)}

    \ti{Struktur och funktion}
    \begin{itemize}
        \ii{ALU (Aritmetisk-logisk enhet) -- utför beräkningar och logiska operationer}
        \ii{Register -- lagrar tillfällig data under exekvering}
        \ii{Kontrollenhet -- styr och koordinerar CPU:ns aktiviteter}
    \end{itemize}

    \blankline
    \ti{Alltså, CPU:n gör:}
    \begin{itemize}
        \ii{Läser instruktioner från minnet}
        \ii{Avkodar (tolkar) dem}
        \ii{Utför instruktionerna (exekverar)}
    \end{itemize}
\end{frame}


\begin{frame}[fragile,t]
    \frametitle{minne i datorer}

    \ti{Typer av minne}
    \begin{itemize}
        \ii{RAM -- arbetsminne som lagrar data och instruktioner temporärt}
        \ii{ROM -- lagrar permanent data, t.ex. firmware}
        \ii{Cache -- snabbare minne nära CPU:n för att optimera åtkomst}
    \end{itemize}

    \blankline
    \ti{Samspel med CPU:n}
    \begin{itemize}
        \ii{CPU:n läser och skriver data till och från RAM}
        \ii{Cache används för att minska åtkomsttiden till ofta använd data}
    \end{itemize}
\end{frame}


\begin{frame}[fragile,t]
    \frametitle{Grundläggande om maskinkod}

    \ti{Definition och egenskaper}
    \begin{itemize}
        \ii{Maskinkod -- binära instruktioner direkt förstådda av CPU:n}
        \ii{Består av enkla operationer som aritmetik, logik, och flytta data}
        \ii{Den specifika instruktionsuppsättningen varierar beroende på CPU}
        \ii{Vanliga instruktioner: \texttt{MOV}, \texttt{ADD}, \texttt{SUB}, \texttt{JMP}}
    \end{itemize}

    \blankline
    \ti{Hur förhåller sig olika ``nivåer'' av språk?}
    \begin{itemize}
        \ii{Maskinkod -- lägst nivå, direkt förstådd av CPU:n}
        \ii{Assemblerspråk -- låg nivå, men använder textrepresentation av instruktioner}
        \ii{Högnivåspråk -- mer abstrakta, översätts till maskinkod via kompilatorer}
    \end{itemize}
\end{frame}


\begin{frame}[fragile,t]
    \frametitle{Maskininstruktioner}

    \ti{komponenter}
    \begin{itemize}
        \ii{Opcode -- specificerar vilken operation som ska utföras}
        \ii{Operand(er) -- data eller adresser som operationen ska tillämpas på}
    \end{itemize}

    \blankline
    \ti{exekvering av instruktioner}
    \begin{itemize}
        \ii{CPU:n hämtar instruktionen från minnet}
        \ii{Avkodar opcode och identifierar operand(er)}
        \ii{Utför operationen och sparar resultatet}
    \end{itemize}

    \blankline
    \ti{exempel}
    \begin{itemize}
        \ii{Exempel skiljer sig mellan olika CPU-arkitekturer}
    \end{itemize}
\end{frame}


\begin{frame}[fragile,t]
    \frametitle{Assembly-språk}


    % Introduction to assembly language as a low-level programming language
    % Relationship between assembly language and machine code
\end{frame}

\begin{frame}[fragile,t]
    \frametitle{Skriva enkel maskinkod}


    % Example of a simple program in machine code
    % Explanation of what each part of the program does
\end{frame}

\begin{frame}[fragile,t]
    \frametitle{Maskinkod och moderna datorer}


    % Relevance of machine code in contemporary software development
    % Overview of how high-level languages are translated into machine code
\end{frame}

\begin{frame}[fragile,t]
    \frametitle{Kompilatorer och tolkar}


    % Explanation of compilers and interpreters
    % How they translate high-level code into executable machine code
\end{frame}

\begin{frame}[fragile,t]
    \frametitle{Utmaningar med att arbeta med maskinkod}


    % Difficulties and limitations of coding in machine code
    % Importance of understanding machine code for debugging and optimization
\end{frame}

\begin{frame}[fragile,t]
    \frametitle{Sammanfattning}


    % Recap of the key points covered
    % Encouragement to explore more about how programming languages work at the machine level
\end{frame}

\begin{frame}[fragile,t]
    \frametitle{Frågor och diskussion}


    % Open the floor for questions
    % Suggest further reading materials and resources
\end{frame}