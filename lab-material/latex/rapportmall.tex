% !TEX encoding = UTF-8 Unicode
\documentclass[a4paper]{article}

% Om du använder pdfLaTeX, inkludera följande två paket, för att få svenska bokstäver och rätt teckenkodning:
\usepackage[T1]{fontenc}      % För svenska bokstäver
\usepackage[utf8]{inputenc}   % Teckenkodning UTF8

% Om du använder XeLaTeX så behöver du inte inkludera ovanstående paket.

% Allt nedanför gäller oavsett om du använder pdfLaTeX eller XeLaTeX

\usepackage[swedish]{babel}   % För svensk avstavning och svenska
                              % rubriker (t ex "Innehållsförteckning")
\usepackage{fancyvrb}         % För programlistor med tabulatorer
\fvset{tabsize=4}             % Tabulatorpositioner
\fvset{fontsize=\small}       % Lagom storlek för programlistor

\title{Dokumentnamn}
\author{Nils Nilsson}
\date{1 augusti 1994}         % Blir dagens datum om det utelämnas

\begin{document}              % Början på dokumentet

\maketitle                    % Skriver ut rubriken som vi
                              % definierade ovan med \title, \author
                              % och eventuellt \date


% Här skrivs all text i dokumentet
MATTIAS -- ÄR  -{}- BÄST --- PÅ -\-- ALLT! \(\sum_{0}^{\infty}1^{\frac{1}{x}} + \ldotp\cdots\ldots\)



\end{document}               % Slut på dokumentet
