% !TEX encoding = UTF-8 Unicode
\documentclass[a4paper]{article}

\usepackage[T1]{fontenc}     % För svenska bokstäver
\usepackage[utf8]{inputenc}  % Teckenkodning UTF8
\usepackage[swedish]{babel}  % För svensk avstavning och svenska
                             % rubriker (t ex "Innehållsförteckning")
\usepackage{fancyvrb}        % För programlistor med tabulatorer
\fvset{tabsize=4}            % Tabulatorpositioner
\fvset{fontsize=\small}      % Lagom storlek för programlistor

\title{Dokumentnamn}
\author{Nils Nilsson}
\date{1 augusti 1994}        % Blir dagens datum om det utelämnas

\begin{document}             % Början på dokumentet

\maketitle                   % Skriver ut rubriken som vi
                             % definierade ovan med \title, \author
                           % och eventuellt \date


ROGER \--\-- ÄR  -{}- BÄST PÅ -\-- ALLT! \--



\end{document}               % Slut på dokumentet
